\documentclass[a4paper,11pt]{article}
%\documentclass[a4paper,11pt]{article}
%\jtwocolumn


\pagestyle{myheadings}\markboth{2-Edge-Connectivity: \today}{2-Edge-Connectivity: \today}

%%%%%%%%%%%%%%%%%%%%%%%%%%%%%%%%%%%%%%%%%%%%%%%%%%%%%%%%%%%%%%%%%%%%%%%%%%%%%%
% packages
\usepackage{graphicx}
\usepackage{epstopdf}
%\usepackage{tabularx}
\usepackage{latexsym}
%\usepackage{pstricks,psfrag}
\usepackage{amsmath,amssymb}
\usepackage{algorithm} %%% for comm. to N-sensei
\usepackage{algpseudocode} %%% for comm. to N-sensei
% \usepackage[ruled,vlined,linesnumbered]{algorithm2e}
\usepackage{tikz}
\usepackage{xcolor}
\usepackage{hyperref}


%%%%%%%%%%%%%%%%%%%%%%%%%%%%%%%%%%%%%%%%%%%%%%%%%%%%%%%%%%%%%%%%%%%%%%%%%%%%%%
% newcommand
%\renewcommand{\mid}{:~}
\newcommand{\secref}[1]{Section~\ref{sec:#1}}
\newcommand{\tabref}[1]{Table~\ref{tab:#1}}
\newcommand{\figref}[1]{Figure~\ref{fig:#1}}
\newcommand{\corref}[1]{Corollary~\ref{cor:#1}}
\newcommand{\lemref}[1]{Lemma~\ref{lem:#1}}
\newcommand{\propref}[1]{Proposition~\ref{prop:#1}}
\newcommand{\thmref}[1]{Theorem~\ref{thm:#1}}

\newcommand{\bbZ}{{\mathbb Z}}
\newcommand{\bbR}{{\mathbb R}}
\newcommand{\MS}{{\mathcal S}}
\newcommand{\MC}{{\mathcal C}}
\newcommand{\MY}{{\mathcal Y}}

\newcommand{\vc}[1]{\mbox{\boldmath$ #1 $}}
\newcommand{\myhline}{
\noindent
\begin{tabular}{p{0.97\textwidth}}
\\\hline
\end{tabular}}
\newcommand{\tab}{\hspace*{1em}}



%%%%%%%%%%%%%%%%%%%%%%%%%%%%%%%%%%%%%%%%%%%%%%%%%%%%%%%%%%%%%%%%%%%%%%%%%%%%%%
% newtheorem
\newtheorem{cor}{Corollary}
\newtheorem{lem}{Lemma}
\newtheorem{prop}{Proposition}
\newtheorem{thm}{Theorem}

%%%%%%%%%%%%%%%%%%%%%%%%%%%%%%%%%%%%%%%%%%%%%%%%%%%%%%%%%%%%%%%%%%%%%%%%%%%%%%
% newenvironment
\newenvironment{proof}{\medskip
  \noindent{\scshape Proof:}}{\quad $\Box$\medskip}

\newenvironment{myframe}{\begin{trivlist}\item[]
    \hrule
    \hbox to \linewidth\bgroup
    \advance\linewidth by -30pt
    \hsize=\linewidth
    \vrule\hfill
    \vbox\bgroup
    \vskip15pt
    \def\thempfootnote{\arabic{mpfootnote}}
    \begin{minipage}{\linewidth}}{%
    \end{minipage}\vskip15pt
    \egroup\hfill\vrule
    \egroup\hrule
\end{trivlist}}

\algrenewcommand{\algorithmicrequire}{\textbf{Input: }}
\algrenewcommand{\algorithmicensure}{\textbf{Output: }}

\newcommand{\algoref}[1]{Algorithm~\ref{alg:#1}}

\long\def\invis#1{}

%%%%%%%%%%%%%%%%%%%%%%%%%%%%%%%%%%%%%%%%%%%%%%%%%%%%%%%%%%%%%%%%%%%%%%%%%%%%%%
% style definitions
%
% following setting makes 3cm spaces for top and bottom, and
% 2.5cm spaces for left and right
%
                                         % default setting
\setlength{\oddsidemargin}{22pt}         % 62pt
\setlength{\evensidemargin}{22pt}        % 62pt
\setlength{\headheight}{12pt}            % 12pt
\setlength{\textheight}{662pt}           % 592pt
\setlength{\marginparsep}{10pt}          % 10pt
\setlength{\footskip}{30pt}              % 30pt
\setlength{\hoffset}{-13pt}              % 0pt
\setlength{\paperwidth}{597pt}           % 597pt
\setlength{\topmargin}{20pt}             % 20pt
\setlength{\headsep}{25pt}               % 25pt
\setlength{\textwidth}{427pt}            % 327pt
\setlength{\marginparwidth}{106pt}       % 106pt
\setlength{\marginparpush}{5pt}          % 5pt
\setlength{\voffset}{-37pt}              % 0pt
\setlength{\paperheight}{845pt}          % 845pt


% 1 inch = 2.54 cm = 72.27 pt

\renewcommand{\baselinestretch}{1.20}
% \setlength{\columnsep}{1.0cm}

\usepackage{enumitem}
\usepackage{xspace}
\usepackage{mfirstuc}
% \newcommand{\lbe}{\mathit{lbe}}
% \newcommand{\lbeTarget}{\mathit{lbeTarget}}
\newcommand{\Cov}{\mathit{Cov}}
\newcommand{\CovMult}{\mathit{CovMult}}
% \newcommand{\CovSrc}{\mathit{CovSrcMult}}
% \newcommand{\SrcExc}{\mathit{SrcExc}}
\newcommand{\Nil}{\textsc{Nil}\xspace}
\newcommand{\True}{\textsc{True}\xspace}
\newcommand{\False}{\textsc{False}\xspace}
\newcommand{\id}{\mathit{id}}
\newcommand{\CovDesc}[1]{\mathit{AggDesc}_{#1}}
\newcommand{\CovAnc}[1]{\mathit{AggAnc}_{#1}}
\newcommand{\Leafset}{\mathit{Leaf}}
\newcommand{\parent}{\pi}
\newcommand{\Childset}{\mathit{Ch}}

\newcommand{\maxTwoDegPath}{degree-two segment\xspace}
\newcommand{\MaxTwoDegPath}{\MFUsentencecase{\maxTwoDegPath}}
\newcommand{\MaxTwoDegPathCap}{Degree-Two Segment\xspace}
\newcommand{\maxTwoDegPaths}{{\maxTwoDegPath}s\xspace}
\newcommand{\MaxTwoDegPaths}{\MFUsentencecase{\maxTwoDegPaths}}
\newcommand{\aggID}{aggregated identifier\xspace}
\newcommand{\AggID}{\MFUsentencecase{\aggID}}
\newcommand{\aggIDs}{{\aggID}s\xspace}
\newcommand{\AggIDs}{\MFUsentencecase{\aggIDs}}



\usepackage[british,american]{babel}
\newif\ifbritish
% \britishtrue   % or false
\makeatletter
\addto\captionsbritish{\britishtrue}
\addto\captionsamerican{\britishfalse}
\makeatother
\newcommand{\engvariation}[2]{%
  \ifbritish
    #1%
  \else
    #2%
  \fi\xspace%
}

\newcommand{\neighbour}[0]{\engvariation{neighbour}{neighbor}}
\newcommand{\Neighbour}[0]{\MFUsentencecase{\neighbour}}
\newcommand{\neighbours}[0]{{\neighbour}s\xspace}
\newcommand{\Neighbours}[0]{\MFUsentencecase{\neighbours}}

\newcommand{\initialise}[0]{\engvariation{initialise}{initialize}}
\newcommand{\Initialise}[0]{\MFUsentencecase{\initialise}}

\newcommand{\categorisation}[0]{\engvariation{categorisation}{categorization}}
\newcommand{\Categorisation}[0]{\MFUsentencecase{\categorisation}}


\newif\ifhyphenationafternon
\hyphenationafternonfalse
\newcommand{\hyphenvariation}[2]{%
  \ifhyphenationafternon
    #1%
  \else
    #2%
  \fi\xspace%
}

\newcommand{\nonzero}{\hyphenvariation{non-zero}{nonzero}}
\newcommand{\nonnegative}{\hyphenvariation{non-negative}{nonnegative}}
\newcommand{\nonempty}{\hyphenvariation{non-empty}{nonempty}}

\newcommand{\plusassign}[2]{#1 := #1 + #2}
\newcommand{\minusassign}[2]{#1 := #1 - #2}
\newcommand{\cupappend}[2]{#1 := #1 \cup #2}
\newcommand{\capappend}[2]{#1 := #1 \cap #2}

\newif\ifsetminusfordiff
\setminusfordifftrue
\newcommand{\diffset}[2]{%
  \ifsetminusfordiff
    #1 \setminus #2%
  \else
    #1 - #2%
  \fi\xspace%
}

%%%%%%%%%%%%%%%%%%%%%%%%%%%%%%%%%%%%%%%%%%%%%%%%%%%%%%%%%%%%%%%%%%%%%%%%%%%%%%
% title information
\title{Simpler Computation of Minimal Removable Sets in 2-Edge-Connected Systems for Undirected Graphs}
\author{SHOTA, Kan\and HARAGUCHI, Kazuya}
%\date{}

%%%%%%%%%%%%%%%%%%%%%%%%%%%%%%%%%%%%%%%%%%%%%%%%%%%%%%%%%%%%%%%%%%%%%%%%%%%%%%
% document
\begin{document}
% \selectlanguage{american}
\selectlanguage{british}
\maketitle

\section{Introduction}
\label{sec:intro}

%%%%%%%%%%%%%%%%%%%%%%%%%%%%%%%%%%%%%%%%%%%%%%%%%%%%%%%%%%%%
\section{Preliminaries}
\label{sec:prel}
\subsection{Graphs}
\label{sec:prel_graphs}
Throughout the paper, we assume that a graph is simple and undirected.
% We also assume that a \nonnegative integer $k$ is given. 
We also assume that each vertex $v$ is associated with a fixed
integer identifier $\id(v)$.
These identifiers are given as part of the input graph
and are preserved in all induced subgraphs.

For a graph $G=(V,E)$ with a vertex set $V$
and an edge set $E$,
we may abbreviate an edge $\{u,v\} \in E$ as $uv$ (or equivalently $vu$)
for simplicity.
% neighbours
For a vertex $v \in V$, we denote by $N_G(v)$ the set of \neighbours of $v$. 
We denote by $N_G(S) := \bigcup_{v \in S} N_G(v) \setminus S$,
the set of open \neighbours of $S$ in $G$.
Let $S \subseteq V$ be a subset of vertices.
For a disjoint subsets $X, Y \subsetneq V$,
$E_G(X,Y) := \{uv \in E \mid u \in X, v \in Y\}$
and when $Y = \diffset{V}{X}$ we simply write $E_G(X)$ instead of $E_G(X, \diffset{V}{X})$.
% induced subgraph
We denote by $G[S]$ the subgraph of $G$
induced by $S$.
We denote by $G-S$ the subgraph obtained by
removing $S$ and all edges to incident to $S$,
that is, $G - S = G[V-S]$.
If $S = \{v\}$,
then we simply write $G-v$ instead of $G - \{v\}$.
%path
A \emph{path} $P$ in $G$ is a sequence of distinct vertices
$(v_1, v_2, \dots, v_k)$
such that $v_i v_{i+1} \in E$ for all $i = 1, 2, \dots, k-1$.
We refer to $v_1$ and $v_k$ as the \emph{endpoint}s of $P$.
% v_1 -> v_2 -> ... -> v_k
We may denote the path $P$ by $v_1 \to v_2 \to \cdots \to v_k$.
% \maxTwoDegPath
A \emph{\maxTwoDegPath} in $G$ is a maximal path
whose vertices all have degree two in $G$.
Note that a \maxTwoDegPath may consist of a single vertex.

\subsection{DFS-Trees and Related Notions}
\label{sec:prel_dfs}
% DFS-Tree
A \emph{DFS-tree} of a connected undirected graph~$G = (V,E)$
is a spanning tree~$T$ obtained by performing a depth-first search (DFS) on $G$.
We root $T$ at an arbitrary vertex $r \in V$.
% Subtree
For each $v\in V$, we denote by $T_v$ the subtree of $T$
rooted at $v$.
% Leaf set
We denote by $\Leafset(T)$ the set of leaves of $T$.

An edge in $E$ is called a \emph{tree edge} if it appears in $T$,
and a \emph{back edge} otherwise.
%
For $u,v \in V$,
we say that $u$ is an \emph{ancestor} of $v$
if $u$ lies on the unique $r$--$v$ path in $T$.
Moreover, if $u$ is an ancestor of $v$ and $u \neq v$,
then we say that $u$ is a \emph{proper ancestor} of $v$,
and $v$ is correspondingly
a (\emph{proper}) \emph{descendant} of $u$.
If $uv$ is a tree edge and $u$ is an ancestor of $v$,
then $u$ is the \emph{parent} of $v$
and $v$ is a \emph{child} of $u$;
we denote this by $\parent(v)=u$.
% children set
We denote by $\Childset(u)$ the set of children of $u$ in $T$.

% % DFS-index
% During the DFS,
% each vertex is assigned a unique integer called \emph{DFS-index},
% which records the order in which the DFS visits the vertices.
% For $v \in V$, we denote its DFS-index by $\mathit{dfsId}(v)$.
% Every vertex has a smaller DFS-index than any of its descendants.

% orientation of edges
We may orient edges based on the DFS-tree $T$:
for a tree edge $\parent(v)v$, we orient it from $\parent(v)$ to $v$; and
for a back edge $xy$, we orient it from the descendant to the ancestor.
We may write $(u,v)$ and $(x,y)$ to indicate the orientation of edge $uv$ and $xy$,
and refer to it as the \emph{outgoing edge} from $u$ or $x$
and the \emph{incoming edge} to $v$ or $y$.

%
We say that a tree edge $\parent(v)v$ is \emph{covered} by a back edge
$e=(x,y)$ if $x$ is a descendant of $v$ and
$y$ is a proper ancestor of $v$
(an ancestor of $\parent(v)$).
In this case, $e$ together with $\parent(v)v$ forms part of a cycle in $G$.
Note that every tree edge is covered by at least one back edge
if $G$ is 2-edge-connected.
For a tree edge $uv$, we denote by
$
\Cov(uv)=\{(x,y)\in E \mid \text{$(x,y)$ is a back edge covering $uv$}\}
$
the set of back edges that cover $uv$.
We refer to $\CovMult(uv) \triangleq |\Cov(uv)|$ as the \emph{cover multiplicity}
of the tree edge $uv$.
% Furthermore, we define the \emph{cover-source multiplicity} of $uv$ to be
% $
% \CovSrc(uv)\triangleq \bigl|\{x \in V(T_v) \mid \text{There is } y \text{ s.t. } (x,y)\in \Cov(uv)\}\bigr|,
% $
% that is, the number of distinct descendants of $v$
% that serve as tails of back edges covering~$uv$.

In addition to cover multiplicity,
we also consider aggregated identifiers of endpoints of covering back edges.
For a tree edge $uv$ and $i \in \{1, 2\}$, we define
\[
\CovDesc{i}(uv) \triangleq \sum_{(x,y) \in \Cov(uv)} \left(\id(x)\right)^i,
\qquad
\CovAnc{i}(uv) \triangleq \sum_{(x,y) \in \Cov(uv)} \left(\id(y)\right)^i.
\]
We refer to these quantities as \emph{\aggIDs}.

% longest back edge outgoing from v
% For a vertex $x \in V$,
% every vertex~$y$ with a back edge $(x, y)$
% lies on the unique $\parent(x)$--$r$ path in $T$.
% Among such vertices~$y$,
% let  $y^*$ denote
% %  the one with the smallest DFS-index,
% % that is, 
% the one closest to the root $r$ in $T$.
% We call the back edge $x y^*$ the \emph{longest back edge outgoing from $x$}
% and denote it by $\lbe(x)$ and $y^*$ by $\lbeTarget(x)$.

\subsection{Graph Contraction and Suppression}
\label{sec:prel_contr_supp}
Let $G=(V,E)$ be a graph.
% contraction
For $S \subseteq V$,
the \emph{contraction} of $S$ in $G$
is the graph obtained by
merging all vertices in $S$ into a single vertex $s$,
removing all edges with both endpoints in $S$,
and replacing each edge with one endpoint in $S$ by an edge incident to $s$.
% suppression
For a vertex $v \in V$ with degree two,
the \emph{suppression} of $v$ in $G$
is the graph obtained by
removing $v$ and its two incident edges,
and adding a new edge between the two \neighbours of $v$.
Equivalently, assuming that $v$ has two \neighbours $u$ and $w$,
we contract the set $\{u,v\}$ or $\{v,w\}$ in $G$.


%%%%%%%%%%%%%%%%%%%%%%%%%%%%%%%%%%%%%%%%%%%%%%%%%%%%%%%%%%%%
\section{Cover Information on DFS-Trees}
\label{sec:cov_info}
% In this section,
% we study structural properties of cover information
% and present an efficient algorithm
% for computing various cover-related quantities.
For later use,
we present efficient algorithms
for computing various cover-related quantities
in this section.
% In \secref{cov_info_struct},
% we establish several structural properties of \aggIDs.
% Although these properties are not immediate from the definitions,
% they play a crucial role in the computation of MinRSs.
%
In \secref{diff_agg},
we introduce a general technique, called \emph{difference-based aggregation},
which efficiently supports path-based update operations on a rooted tree.
%
In \secref{cov_info_agg},
we apply this technique to compute various cover-related quantities
for all tree edges in a 2-edge-connected undirected graph
in $O(n+m)$ time.

Before presenting the algorithms,
we briefly illustrate how \aggIDs
can be used to detect structural properties
of covering back edges.
%
For a tree edge $uv$,
we can determine whether all covering back edges are
outgoing from the same descendant or
incoming to the same ancestor
using the \aggIDs.
%
In order to show this,
we first prove a general lemma on sequences of real numbers.

\begin{lem}
  \label{lem:seq_all_equal}
  Let $(\alpha_1, \alpha_2, \dots, \alpha_q)$ be a sequence of $q$ real numbers.
  Then, all elements in the sequence are equal, i.e.,
  $\alpha_1 = \alpha_2 = \cdots = \alpha_q$,
  if and only if
  \[
  \left(\sum_{i=1}^{q} \alpha_i\right)^2 = q  \sum_{i=1}^{q} \left(\alpha_i\right)^2.
  \]
\end{lem}
\begin{proof}
  We have
  \begin{align*}
    &q \sum_{i=1}^{q} \left(\alpha_i\right)^2
    - \left(\sum_{i=1}^{q} \alpha_i\right)^2
    = \sum_{i=1}^{q} (q-1)\left(\alpha_i\right)^2
    - \sum_{i=1}^{q-1} \sum_{j=i+1}^{q} 2 \alpha_i \alpha_j 
    \\
    &= \sum_{i=1}^{q} (q-i + i -1)\left(\alpha_i\right)^2
    - \sum_{i=1}^{q-1} \sum_{j=i+1}^{q} 2 \alpha_i \alpha_j
    \\
    &= \sum_{i=1}^{q-1} \sum_{j=i+1}^{q} \left(\alpha_i\right)^2
    + \sum_{i=2}^{q} \sum_{j=1}^{i-1} \left(\alpha_i\right)^2
    - \sum_{i=1}^{q-1} \sum_{j=i+1}^{q} 2 \alpha_i \alpha_j
    \\
    &= \sum_{i=1}^{q-1} \sum_{j=i+1}^{q} \left[
      \left(\alpha_i\right)^2 - 2 \alpha_i \alpha_j + \left(\alpha_j\right)^2 
    \right]
    = \sum_{i=1}^{q-1} \sum_{j=i+1}^{q} \left(\alpha_i - \alpha_j\right)^2.
  \end{align*}
  The entire summation is equal to zero if and only if
  $\alpha_i = \alpha_j$ for all $i,j \in \{1, \dots, q\}$
  since each term $\left(\alpha_i - \alpha_j\right)^2$ is \nonnegative.
\end{proof}

Using \lemref{seq_all_equal},
we can prove the following lemma
on \aggIDs.
\begin{lem}
  \label{lem:cov_info_use}
  Let $G=(V,E)$ be a 2-edge-connected undirected graph,
  $T$ be a DFS-tree of $G$ rooted at $r \in V$,
  and $uv$ be a tree edge in $T$.
  Moreover,
  let $\Cov(uv) = \{(x_1,y_1), \dots, (x_q,y_q)\}$ be the set of back edges covering $uv$,
  where $q = \CovMult(uv)$.
  Then, the following two conditions hold:
  \begin{itemize}
  \item $x_1 = x_2 = \cdots = x_q$ if and only if
    $\left(\CovDesc{1}(uv)\right)^2 = q \cdot \CovDesc{2}(uv)$; and
  \item $y_1 = y_2 = \cdots = y_q$ if and only if
    $\left(\CovAnc{1}(uv)\right)^2 = q \cdot \CovAnc{2}(uv)$.
  \end{itemize}
\end{lem}
\begin{proof}
  The conditions follow directly from \lemref{seq_all_equal}
  by setting
  $\alpha_i := \id(x_i)$ for the descendant-side condition
  and $\alpha_i := \id(y_i)$ for the ancestor-side condition.
\end{proof}


\subsection{Difference-Based Aggregation on a Rooted Tree}
\label{sec:diff_agg}
Let $T$ be a rooted tree.
We let each edge $e$ store a value, say $A(e)$
(from $\mathbb{Z}$, $\mathbb{R}$,
or, more generally, from an additive abelian group),
and consider update operations
that add a constant value
to all edges on a specified
ancestor--descendant path in $T$.

Formally, update operations are specified by triples $(x, y,\alpha)$,
where $y$ is an ancestor of $x$ in $T$,
and $\alpha$ is a value to be added to all edges on the $y$--$x$ path.
The goal is to compute the final value stored in each edge
after performing a sequence of such update operations.

In order to achieve this goal efficiently,
we use a difference-based approach.
Specifically, we introduce an auxiliary value $\Delta(v)$
for each vertex $v$ in $T$,
which will be used to represent the effect of update operations.

\begin{lem}
  \label{lem:diff_agg}
  Let $T$ be a rooted tree.
  Each edge $e$ of $E$ stores an initial value $A(e)$ from an additive abelian group.
  For each vertex $v$ in $T$,
  let $\Delta(v)$ be an auxiliary value,
  initially set to zero.

  For each update operation $(x,y,\alpha)$,
  where $y$ is an  ancestor of $x$ in $T$,
  we perform:
  \[
    \plusassign{\Delta(x)}{\alpha},
    \qquad
    \minusassign{\Delta(y)}{\alpha}.
  \]

  Then, after processing a sequence of update operations,
  for each edge $\parent(v)v$, 
  the updated value is given by
  \[
  A(\parent(v)v) + \sum_{w \in V(T_v)} \Delta(w).
  \]
\end{lem}
\begin{proof}
  It suffices to show the claim for a single update operation $(x,y,\alpha)$
  since the contributions of multiple operations are additive.
  By the definition of $\Delta$,
  the operation contributes $+\alpha$ to $\Delta(x)$ and $-\alpha$ to $\Delta(y)$.
  We examine how such an operation contributes to the sum~$\sum_{w \in V(T_v)} \Delta(w)$.

  If $x \in V(T_v)$ and $y \not\in V(T_v)$,
  then the $x$--$y$ path passes through $\parent(v)v$,
  and the operation contributes $+\alpha$ to the sum.
  If both $x$ and $y$ belong to $V(T_v)$,
  then the $x$--$y$ path does not pass through $\parent(v)v$,
  and its $+\alpha$ and $-\alpha$ contributions cancel within $T_v$,
  yielding a total contribution of $0$.
  If neither $x$ nor $y$ is in $T_v$,
  then the $x$--$y$ path does not pass through $\parent(v)v$,
  and again contributes $0$ to the sum.
  The case with $x \not\in V(T_v)$ and $y \in V(T_v)$ cannot occur 
  since $y$ is a proper ancestor of $x$.

  Therefore, the only update operations that contribute a \nonzero amount are those
  whose $x$--$y$ path passes through $\parent(v)v$,
  and each such operation contributes exactly $+\alpha$.
\end{proof}

\begin{algorithm}[t]
  \caption{Difference-Based Aggregation on a Rooted Tree}
  \label{alg:diff_agg}
  \begin{algorithmic}[1]
    \Require A rooted tree $T$,
    an initial value $A(e)$ for all edges $e$ in $E(T)$,
    and a sequence of update operations $(x,y,\alpha)$
    \Ensure The final values for all edges
    % % \State \Initialise $\Delta(v) := 0$ and $A(e) := 0$
    % for all $v \in V(T)$ and $e \in E(T)$;
    \State \Initialise $\Delta(v) := 0$ for all $v \in V(T)$ and $\mathit{diff}(e) := 0$ for all $e \in E(T)$;
    \For{\textbf{each} update operation $(x,y,\alpha)$}
      \State $\plusassign{\Delta(x)}{\alpha}$;
      $\minusassign{\Delta(y)}{\alpha}$
      \label{codeline:diff_agg_update}
    \EndFor;
    \For{\textbf{each} child $w \in \Childset(r)$ of the root $r$ in $T$}
      \State Execute \Call{Aggregate-DFS}{$w$}
    \EndFor;
    \State \textbf{output} $A(e) + \mathit{diff}(e)$ for all edges $e$ in $E(T)$

    \\\Procedure{Aggregate-DFS}{$v$}
    \State $\mathit{diff}(\parent(v)v) := \Delta(v)$;
    \label{codeline:diff_agg_diff_update_one_Delta}
    \For{\textbf{each} child $w \in \Childset(v)$ in $T$}
      \State Execute \Call{Aggregate-DFS}{$w$};
      \State $\plusassign{\mathit{diff}(\parent(v)v)}{\mathit{diff}(vw)}$
    \EndFor\label{codeline:diff_agg_diff_update_children_end}
    \EndProcedure
  \end{algorithmic}
\end{algorithm}

\begin{lem}
  \label{lem:diff_agg_algo}
  Given a rooted tree $T$ with $n$ vertices,
  initial values $A(e)$ for all edges $e$ in $E(T)$,
  and a sequence of $q$ update operations,
  \algoref{diff_agg} computes the final values
  for all edges in $O(q+n)$ time.
\end{lem}
\begin{proof}
  By \lemref{diff_agg},
  it suffices to show that
  \algoref{diff_agg} correctly computes
  $
  A(\parent(v)v) + \sum_{w \in V(T_v)} \Delta(w)
  $
  for each edge $\parent(v)v$.
  % the sum of the auxiliary values $\Delta$ for all vertices in $T_v$.
  For $\Delta$, line~\ref{codeline:diff_agg_update}
  is identical to that in \lemref{diff_agg}.

  Thus, it remains to show that
  \Call{Aggregate-DFS}{$v$} correctly computes
  $\mathit{diff}(\parent(v)v) = \sum_{w \in V(T_v)} \Delta(w)$
  for each non-root vertex $v$ in $T$.
  We first show the case with $v$ being a leaf.
  In this case, 
  the only coputation in \Call{Aggregate-DFS}{$v$} is in line~\ref{codeline:diff_agg_diff_update_one_Delta}
  since there is no child of $v$ in $T$;
  we have $\mathit{diff}(\parent(v)v) = \Delta(v)$.
  
  Next, we consider the case other than the leaf case. 
  We assume that, for any child $w$ of $v$ in $T$,
  \Call{Aggregate-DFS}{$w$} correctly computes $\mathit{diff}(vw) = \sum_{z \in V(T_w)} \Delta(z)$,
  the sum of the auxiliary values $\Delta$ for all vertices in $T_w$.
  Then, lines~\ref{codeline:diff_agg_diff_update_one_Delta}--\ref{codeline:diff_agg_diff_update_children_end} compute
  \[
  \mathit{diff}(\parent(v)v)
  =
  \Delta(v) + \sum_{w \in \Childset(v)} \mathit{diff}(vw)
  =
  \Delta(v) + \sum_{w \in \Childset(v)} \sum_{z \in V(T_w)} \Delta(z)
  = \sum_{z \in V(T_v)} \Delta(z).
  \]

  Each update operation is processed in constant time,
  which means that
  the total time for all $q$ update operations is $O(q)$.
  We visit each vertex exactly once,
  and consider all edges exactly once,
  which means we can correctly compute the final values for all edges.
  We can execute the aggregation in $O(n)$ time
  since we traverse only edges in $T$.
\end{proof}

\subsection{Computation of Cover Information}
\label{sec:cov_info_agg}
Let $G=(V,E)$ be a 2-edge-connected undirected graph 
and $T$ be a DFS-tree of $G$ rooted at $r \in V$.
As described in \secref{prel_dfs},
every back edge $(x,y)$ is oriented from a descendant $x$ to a proper ancestor $y$.

Each back edge $(x,y)$ covers
all tree edges on the unique $y$--$x$ path in $T$.
Naturally, we have a mapping from back edges
to an update operation $(x,y,\alpha)$ on $T$,
where $\alpha$ is chosen appropriately depending on the quantity to be computed.

\begin{lem}
  \label{lem:cov_info_agg}
  Given a 2-edge-connected undirected graph $G=(V,E)$
  with $n=|V|$ and $m=|E|$,
  and a DFS-tree $T$ of $G$ rooted at $r \in V$,
  we can compute the following values for all tree edges in $T$
  in $O(n+m)$ time:
  \begin{itemize}
  \item Cover multiplicity $\CovMult(uv)$;
  % \item Cover-source multiplicity $\CovSrc(uv)$;
  \item \AggIDs $\CovDesc{1}(uv)$ and $\CovDesc{2}(uv)$; and
  \item \AggIDs $\CovAnc{1}(uv)$ and $\CovAnc{2}(uv)$.
  \end{itemize}
\end{lem}
\begin{proof}
  By the definiton of each quantity,
  we can represent the contribution of each back edge $(x,y)$
  as an update operation $(x,y,\alpha)$ on $T$,
  where $\alpha$ is set as follows:
  \begin{itemize}
  \item For $\CovMult(uv)$, set $\alpha := 1$;
  % \item For $\CovSrc(uv)$, set $\alpha := 1$ if $(x, y) = \lbe(x)$; otherwise, set $\alpha := 0$;
  \item For $\CovDesc{i}(uv)$, set $\alpha := \left(\id(x)\right)^i$ for $i \in \{1, 2\}$; and
  \item For $\CovAnc{i}(uv)$, set $\alpha := \left(\id(y)\right)^i$ for $i \in \{1, 2\}$.
  \end{itemize}
  There are $O(m)$ back edges in $G$, 
  so
  we can perform $O(m)$ update operations on $T$.
  By \lemref{diff_agg_algo},
  we can compute the final values for all tree edges in $O(n+m)$ time.
\end{proof}

%%%%%
% \section{Computation of Cover multiplicities}
% \label{sec:cov_mult}
% In this section, we present an $O(n+m)$-time algorithm
% to compute the cover multiplicity and the cover-source multiplicity
% for all tree edges in a 2-edge-connected undirected graph $G=(V,E)$
% with $n=|V|$ and $m=|E|$.

% Since the two multiplicities can be computed in a similar manner,
% we here describe the algorithm for computing the cover multiplicities;
% the cover-source multiplicities can be computed with a minor modification
% as described at the end of this section.

% In this section, we assume that a DFS-tree $T$ of $G$,
% rooted at an arbitrary vertex $r \in V$, is already given.
% The algorithm consists of two phases.
% In the first phase, we process each back edge to update an auxiliary
% value called \emph{cover-delta value} associated with its endpoints.
% %(Algorithm~\ref{alg:preproc_backedge}).
% In the second phase, we traverse $T$ in a depth-first manner to aggregate these values
% along the tree and compute the multiplicity for each tree edge
% (Algorithm~\ref{alg:cov_mult}).
% % The details are described below.

% In order to compute the cover multiplicities,
% we define the \emph{cover-delta value} $\delta(v)$ for each vertex $v \in V$ as follows:
% \[
% \delta(v)
% \triangleq \big| \{(v,y) \mid (v,y) \text{ is a back edge}\} \big| - \big| \{(x,v) \mid (x,v) \text{ is a back edge}\} \big|.
% \]
% Equivalently, 
% if we orient all back edges from descendants to ancestors as described in \secref{prel_dfs}
% and remove all tree edges,
% then $\delta(v)$ equals the outdegree minus the indegree of $v$ in the resulting directed graph.
% The cover-delta values satisfy a fundamental relation with the cover
% multiplicity, which we show below.

% \begin{lem}
%   \label{lem:delta_cov}
%   For any tree edge $(u,v)$ of $G$,
%   the cover multiplicity of $(u,v)$ is equal to the sum of
%   the cover-delta values for all vertices in $T_v$:
%   \[
%   \CovMult(uv) = \sum_{w \in V(T_v)} \delta(w).
%   \]
% \end{lem}
% \begin{proof}
%   By the definition of $\delta$,
%   every back edge $(x,y)$ contributes $+1$ to $\delta(x)$ and $-1$ to $\delta(y)$.
%   We examine how such a back edge contributes to the sum~$\sum_{w \in V(T_v)} \delta(w)$.

%   If $x \in V(T_v)$ and $y \not\in V(T_v)$,
%   then $(x,y)$ covers $uv$ and contributes $+1$ to the sum.
%   If both $x$ and $y$ belong to $V(T_v)$,
%   then $(x,y)$ does not cover $uv$ and
%   its $+1$ and $-1$ contributions cancel within $T_v$,
%   yielding a total contribution of $0$.
%   If neither $x$ nor $y$ is in $T_v$,
%   then $(x,y)$ does not cover $uv$ and again contributes $0$ to the sum.
%   Since $y$ is a proper ancestor of $x$,
%   the case with $x \not\in V(T_v)$ and $y \in V(T_v)$ cannot occur.

%   Therefore, the only back edges that contribute a \nonzero amount are those
%   that cover $uv$,
%   and each such edge contributes exactly $+1$.
% \end{proof}

% In order to compute the cover-delta values for all vertices in $V$,
% we can process each back edge $(x,y)$ of $G$
% and increment $\delta(x)$ by 1 and decrement $\delta(y)$ by 1.
% This can be done in $O(n+m)$ time
% by performing a DFS on $G$ to obtain its DFS-tree $T$
% and then processing all back edges of $G$.

% The second phase computes the cover multiplicities for all tree edges
% by aggregating the cover-delta values along the DFS-tree $T$.
% The details are shown in Algorithm~\ref{alg:cov_mult}.


% % Algo I
% \begin{algorithm}[t]
%   \caption{Computation of Cover Multiplicities by DFS}
%   \label{alg:cov_mult}
%   \begin{algorithmic}[1]
%     \Require A 2-edge-connected undirected graph $G=(V,E)$,
%     a DFS-tree $T$ of $G$ rooted at $r \in V$,
%     and
%     the cover-delta values $\delta$ for all vertices in $V$
%     \Ensure The cover multiplicities of all tree edges
%     \State $\CovMult := {}$ an array from tree edges to integers;
%     \For{\textbf{each} child $w$ of the root $r$ in $T$}
%       \State Execute \Call{CoverMultiplicity-DFS}{$w$}
%     \EndFor;
%     \State \textbf{output} $\CovMult(e)$ for all tree edges $e$ in $E(T)$
    
%     \\\Procedure{CoverMultiplicity-DFS}{$v$}
%     \State $\CovMult(\parent(v)v) := \delta(v)$;
%     \For{\textbf{each} child $w$ of $v$ in $T$}
%       \State Execute \Call{CoverMultiplicity-DFS}{$w$};
%       \State $\CovMult(\parent(v)v) := \CovMult(\parent(v)v) + \CovMult(vw)$
%     \EndFor
%     % \State \textbf{return} $\CovMult(uv)$;
%     \EndProcedure
%   \end{algorithmic}
% \end{algorithm}

% % % Algo II
% % \begin{algorithm}[t]
% %   \caption{Preprocessing Back Edges}
% %   \label{alg:preproc_backedge}
% %   \begin{algorithmic}[1]
% %     \Require A 2-edge-connected undirected graph $G=(V,E)$
% %     and its DFS-tree $T$ rooted at $r \in V$
% %     \Ensure The cover-delta values $\delta$ for all vertices in $V$
% %     \State Initialize $\delta(v) := 0$ for all $v \in V$;
% %     \For{\textbf{each} back edge $(x,y)$ of $G$}
% %     \label{codeline:backedge_loop}
% %       \State $\delta(x) := \delta(x) + 1$;
% %       \State $\delta(y) := \delta(y) - 1$
% %     \EndFor;
% %     \State \textbf{return} $\delta$
% %   \end{algorithmic}
% % \end{algorithm}

% % The overall algorithm is as follows:
% % \begin{enumerate}
% % \item Execute Algorithm~\ref{alg:preproc_backedge} to compute
% %   the cover-delta values for all vertices in $V$.
% % \item Execute Algorithm~\ref{alg:cov_mult} with the tree edge
% %   incident to the root $r$ to compute the cover multiplicities
% %   for all tree edges in $T$.
% % \end{enumerate}


% \begin{lem}
%   \label{lem:cov_mult}
%   Given a 2-edge-connected undirected graph $G=(V,E)$
%   with $n=|V|$ and $m=|E|$,
%   its DFS-tree $T$ rooted at $r \in V$, and 
%   the cover-delta values $\delta$ for all vertices in $V$,
%   Algorithm~\ref{alg:cov_mult} computes the cover multiplicities
%   for all tree edges in $O(n)$ time.
% \end{lem}
% \begin{proof}
%   By \lemref{delta_cov},
%   it suffices to show that
%   Algorithm~\ref{alg:cov_mult} correctly computes
%   the sum of the cover-delta values for all vertices in $T_v$.

%   We first show the case with $v$ being a leaf.
%   In this case, since there is no child of $v$ in $T$,
%   the algorithm returns $\CovMult(uv) = \delta(v)$.
  
%   Next, we consider the case other than the leaf case. 
%   We assume that, for any child $w$ of $v$ in $T$,
%   \Call{CoverMultiplicity-DFS}{$w$} correctly computes
%   the sum of the cover-delta values for all vertices in $T_w$.
%   Then, the algorithm computes
%   \[
%   \delta(v) + \sum_{w \text{:~child of } v} \sum_{z \in V(T_w)} \delta(z)
%   = \sum_{z \in V(T_v)} \delta(z)
%   \]
%   that is the sum of the cover-delta values for all vertices in $T_v$.

%   We visit each vertex exactly once,
%   and consider all tree edges exactly once,
%   which means we can correctly compute the cover multiplicities for all tree edges.
%   Since we traverse only tree edges in $T$,
%   we can execute the algorithm in $O(n)$ time.
% \end{proof}

% % If we want to compute the cover-source multiplicities,
% % we can modify Algorithm~\ref{alg:preproc_backedge}
% % so that, in line \ref{codeline:backedge_loop} of Algorithm~\ref{alg:preproc_backedge},
% % we only consider the longest back edges outgoing from each vertex.
% To compute the cover-source multiplicities,
% we modify the definition of the cover-delta value so that
% it accounts only for the longest outgoing back edges.
% We define the \emph{cover-source-delta} value $\delta_{\text{src}}(v)$
% to be
% \[
% \delta_{\text{src}}(v)
% \triangleq
% \left.\begin{cases}
% 1 & \text{if } \lbe(v) \text{ exists} \\
% 0 & \text{otherwise}
% \end{cases}
% \right\}
% % \big| \{\lbeTarget(x) \mid \lbe(x) \text{ exists}\} \big|
% - \big| \{x \in V \mid \lbe(x) = (x,v)\} \big|.
% \]
% Equivalently,
% if we orient only the longest back edges from descendants to ancestors
% and remove all the other edges,
% then $\delta_{\text{src}}(v)$ equals the outdegree minus the indegree of $v$ in the resulting directed graph.
% Then, by a similar argument to \lemref{delta_cov},
% we can show that
% \[
% \CovSrc(uv) = \sum_{w \in V(T_v)} \delta_{\text{src}}(w).
% \]
% Thus, by replacing $\delta$ with $\delta_{\text{src}}$ in Algorithm~\ref{alg:cov_mult},
% we can compute the cover-source multiplicities for all tree edges.


% Therefore, we obtain the following theorem that 
% \engvariation{summarises}{summarizes} the results in this section.

% \begin{thm}
%   \label{thm:cov_mult}
%   Given a 2-edge-connected undirected graph $G=(V,E)$
%   with $n=|V|$ and $m=|E|$,
%   we can compute the cover multiplicity and the cover-source multiplicity
%   for all tree edges in $G$ in $O(n+m)$ time.
% \end{thm}
% \begin{proof}
%   We first perform a DFS on $G$ to obtain its DFS-tree $T$
%   rooted at an arbitrary vertex $r \in V$ in $O(n+m)$ time.
%   Next, we process all back edges of $T$
%   to compute the cover-delta values~$\delta$ or cover-source-delta values~$\delta_{\text{src}}$ for all vertices in $V$ in $O(n+m)$ time.
%   Finally, we execute Algorithm~\ref{alg:cov_mult}
%   with each tree edge $(r, v)$ incident to the root $r$
%   to compute the cover multiplicities for all tree edges in $T$.
%   By \lemref{cov_mult},
%   this step takes $O(n)$ time.
%   Thus, the total time complexity is $O(n+m)$.
% \end{proof}

%%%%%%%%%%%%%%%%%%%%%%%%%%%%%%%%%%%%%%%%%%%%%%%%%%%%%%%%%%%%
\section{Mathematical Properties and Computation of Minimal Removable Sets}
\label{sec:mrs_props}
In this section, we study structural properties of MinRSs
and present an efficient algorithm for computing all MinRSs
in a 2-edge-connected undirected graph.
% In \secref{mrs_props},


MinRSs in a 2-edge-connected undirected graph
have been studied in \cite{ISYH.2022,TH.2025}.
In these works, they showed the following necessary condition for MinRSs.
\begin{lem}[Observation 3 in \cite{ISYH.2022} and Lemma 7 in \cite{TH.2025}]
  \label{lem:mrs_cond}
  Let $G=(V,E)$ be a 2-edge-connected undirected graph with $|V|\geq 3$.
  If a \nonempty proper subset $S \subsetneq V$ is a MinRS of $G$,
  then, either
  \begin{itemize}
  \item $S$ forms a \maxTwoDegPath in $G$;
  or
  \item $S$ is a singleton $\{v\}$, where $v \in V$ and $\deg_G(v) \geq 3$.
  \end{itemize}
\end{lem}
By degree conditions,
this \categorisation is mutually exclusive.
By \lemref{mrs_cond},
we can list all candidates for MinRSs in $O(n+m)$ time
by finding all \maxTwoDegPaths in $G$
and collecting all singleton vertex sets.

% Moreover,
% if $r$ is contained in some MinRS,
% then it is the only such MinRS in $G$:
% $\{r\}$ if $\deg_G(r) \geq 3$ or a \maxTwoDegPath containing $r$ otherwise.


\begin{lem}
  \label{lem:mrs_root}
  Let $G=(V,E)$ be a 2-edge-connected undirected graph with $|V|\geq 3$
  and $T$ be a DFS-tree of $G$ rooted at $r\in V$.
  We can determine whether $G$ has a MinRS containing $r$
  and find such a MinRS if exists
  in $O(n+m)$ time.
\end{lem}
\begin{proof}
  By \lemref{mrs_cond},
  if $G$ has a MinRS containing $r$,
  then it is either $\{r\}$ (if $\deg_G(r) \geq 3$) or a \maxTwoDegPath~$P$ containing $r$ (otherwise).
  All we have to do is to check $G - r$ or $G - P$ for 2-edge-connectivity, which can be done in $O(n+m)$ time.
\end{proof}

In the rest of this section,
we focus on MinRSs that do not contain the root of the DFS-tree.
%
Before going into details,
we show a basic property of \maxTwoDegPaths not containing the root of the DFS-tree
for the sake of simplifying discussions in the following subsections.
\begin{lem}
  \label{lem:max2degpath_props}
  Let $G=(V,E)$ be a 2-edge-connected undirected graph with $|V|\geq 3$,
  $T$ be a DFS-tree of $G$ rooted at $r\in V$, and
  Let $P = v_1 \to v_2 \to \cdots \to v_k$
  be a \maxTwoDegPath in $G$ that does not contain $r$.
  Then, the following hold:
  \begin{enumerate}[label={\rm(\roman*)}]
  \item
  no internal vertex of $T$
  that is an endpoint of any back edge
  appears in $P$; and
  % \item The sequence $v_1, v_2, \ldots, v_k$ appears in this order or its reverse order
  % on a root-to-leaf path in $T$.
  \item exactly one of $v_1 \to v_2 \to \cdots \to v_k$
    and $v_k \to v_{k-1} \to \cdots \to v_1$
    is a path on $T$.
  \end{enumerate}
\end{lem}
\begin{proof}
  (i)
  Every internal vertex of $T$ has the parent and at least one child.
  If an internal vertex of $T$ is an endpoint of some back edge,
  then the degree in $G$ is at least three.
  Thus, such a vertex cannot appear in $P$.
  \newline
  (ii)
  By (i),
  all vertices in $P$ other than $v_1$ and $v_k$
  has no \neighbours outside $P$ in $G$.
  Thus, in the DFS traversal that constructs $T$,
  the vertices in $P$ are visited one by one
  without visiting any other vertex in between.
  % \newline
  % (iii) is immediate from (ii).
\end{proof}

Based on \lemref{mrs_cond},
we classify MinRSs into several types
to make them easier to handle with a DFS-tree $T$ of $G$ rooted at an arbitrary vertex $r \in V$:
\begin{itemize}
\item a MinRS containing the root of the DFS-tree
(unique if exists);
\item MinRSs containing exactly one leaf of the DFS-tree;
\item MinRSs consisting only of internal vertices of the DFS-tree that form a \maxTwoDegPath in $G$; and
\item MinRSs consisting of a single internal vertex of the DFS-tree with degree at least three in $G$.
\end{itemize}

If all the vertices in the root-to-leaf path in $T$
has degree two in $G$,
then $G$ is a cycle and has no MinRS.
Otherwise, an MinRS containing the root of the DFS-tree
contains no leaf of the DFS-tree.
By \lemref{max2degpath_props}(ii),
a MinRS containing at most one leaf of the DFS-tree.
Thus,
this \categorisation is mutually exclusive and exhaustive.

\subsection{MinRSs Containing DFS-Tree Leaf}
\label{sec:leaf_mrs}
In this subsection, we consider MinRSs containing exactly one leaf of a DFS-tree.
Note that \maxTwoDegPath can be a single vertex,
which means this path consists of only a leaf of the DFS-tree.

We first show a basic property of leaves in a DFS-tree
of a 2-edge-connected undirected graph.
\begin{lem}
  \label{lem:leaf_not_ap}
  Let $G=(V,E)$ be a 2-edge-connected undirected graph with $|V|\geq 3$
  and
  $T$ be a DFS-tree of $G$ rooted at $r\in V$.
  For any leaf $v\in \Leafset(T)$ or
  any \maxTwoDegPath~$P$ containing a leaf of $T$,
  $G - v$ or $G - P$ is connected.
\end{lem}
\begin{proof}
  Every path in $T$ also exists in $G$
  since $T$ is a spanning tree of $G$.
  For any leaf $v\in \Leafset(T)$ or
  any \maxTwoDegPath~$P$ containing a leaf of $T$,
  $T - v$ or $T - P$ is connected.
  Thus, $G - v$ or $G - P$ is also connected.
\end{proof}

% \begin{lem}
%   \label{lem:leaf_mrs}
%   Let $G=(V,E)$ be a 2-edge-connected undirected graph with $|V|\geq 3$ and,
%   $T$ be a DFS-tree of $G$ rooted at $r\in V$.
%   For each leaf $v$, $\{v\}$ is a MinRS of $G$ if and only if
%   the following two conditions hold:
%   \begin{itemize}
%   \item[\rm (i)] $\parent(v)$ has degree at least 3 in $G$; and
%   % \item[\rm (ii)] for any back edge $vu$,
%   % any tree edge on the $u,\parent(v)$-path in $T$ is covered by
%   % some back edge outgoing from other vertex than $v$. 
%   \item[\rm (ii)] let $\lbe(v)=(v,u^*)$,
%     then, for any tree edge $(x,y)$ on the $u^*,\parent(v)$-path in $T$,
%     there exists a back edge outgoing from a vertex in $T_y - v$
%     that covers $(x,y)$.
%   \end{itemize}
% \end{lem}
% \begin{proof}
%   ($\Longrightarrow$)
%   We show by contraposition.
%   Suppose that either (i) or (ii) does not hold.
%   If (i) does not hold, then
%   $\parent(v)$ has degree 1 in $G - v$,
%   which means that $G - v$ is not 2-edge-connected.
%   If (ii) does not hold,
%   then there exists a tree edge $(x,y)$ on the $u^*,\parent(v)$-path in $T$
%   that is not covered by any back edge outgoing from other vertex than $v$.
%   Since any vertex in $T_y - v$ cannot have a back edge to
%   a proper ancestor of $x$ in $G - v$,
%   which means that $xy$ is a bridge in $G - v$.
%   Thus, in either case, $G - v$ is not 2-edge-connected.
  
%   % \medskip
%   ($\Longleftarrow$)
%   We suffice to show that $G - v$ is 2-edge-connected.
%   By \lemref{leaf_not_ap},
%   $G - v$ is connected.
%   We show that there is no bridge in $G - v$.
%   Every back edge $(x, y)$ with $x, y \neq v$ is not a bridge in $G - v$,
%   since $xy$ and the $y,x$-path in $T$ form a cycle in $G - v$.
%   Next, we consider tree edges in $G - v$.
%   For any tree edge not on the $u^*,\parent(v)$-path in $T$,
%   the covering back edge still exists in $G - v$,
%   which means that such a tree edge is not a bridge in $G - v$.
%   For any tree edge $(x,y)$ on the $u^*,\parent(v)$-path in $T$,
%   by (ii),
%   there exists a back edge outgoing from a vertex in $T_y - v$
%   that covers $(x,y)$,
%   which means that such a tree edge is not a bridge in $G - v$.
%   Thus, there is no bridge in $G - v$,
%   and hence, $G - v$ is 2-edge-connected.
% \end{proof}


\begin{lem}
  \label{lem:leaf_mrs}
  Let $G=(V,E)$ be a 2-edge-connected undirected graph with $|V|\geq 3$.
  Let $T$ be a DFS-tree of $G$ rooted at $r\in V$
  and let $x \in \Leafset(T)$.
  Let $S \subsetneq V$ be 
  either
  a nonempty proper subset of $V$ that
  forms the \maxTwoDegPath that visits $x$
  when $\deg_G(x) = 2$;
  or $S = \{x\}$ when $\deg_G(x) \geq 3$.
  %
  The following are equivalent:
  \begin{enumerate}[label={\rm(\roman*)}]
    \item $S$ is a MinRS of $G$;
    \item for any tree edge $(u,v)$ with $v \not\in S$,
      there exists a back edge outgoing from some vertex in $\diffset{V}{S}$
      that covers $(u,v)$; and
    \item for any tree edge $(u,v)$ with $v \not\in S$,
      there exists a back edge outgoing from some vertex in $\diffset{V}{\{x\}}$
      that covers $(u,v)$.
  \end{enumerate}
% 
  % Then, $S$ is a MinRS of $G$ if and only if,
  % for any tree edge $(u,v)$ with $v \not\in S$,
  % there exists a back edge outgoing from some vertex in $\diffset{V}{S}$
  % that covers $(u,v)$.  
\end{lem}
\begin{proof}
  By \lemref{max2degpath_props}(ii),
  $S$ forms a path in $T$, which leads to that
  $S$ contains exactly one leaf of $T$ and
  $S$ appears as a path from some vertex~$a$ to its descendant~$x$ in $T$.
  (If $S$ is a singleton, then $a = x$.)
  By the definition of $S$, $x$ is a leaf of $T$.
  % Since $G$ is 2-edge-connected,
  Every tree edge is covered by at least one back edge
  due to the 2-edge-connectivity of $G$.
  Thus, it suffices to consider only tree edges on the $y$--$\parent(a)$ path in $T$,
  where $y$ is the most distant proper ancestor of $x$
  among those adjacent to $x$ by a back edge.
  \newline
  ($\text{(i)}\Longrightarrow\text{(ii)}$)
  We show by contraposition.
  Suppose that there exists a tree edge $(u,v)$ on the $y$--$\parent(a)$ path in $T$
  that is not covered by any back edge outgoing from a vertex  other than those in $S$.
  Any vertex in $T_v - S$ cannot have a back edge to
  a proper ancestor of $u$ in $G - S$,
  which means that $uv$ is a bridge in $G - S$.
  \newline
  ($\text{(ii)}\Longrightarrow\text{(i)}$)
  By \lemref{mrs_cond},
  it suffices to show that $G - S$ is 2-edge-connected.
  By \lemref{leaf_not_ap},
  $G - S$ is connected.
  We show that there is no bridge in $G - S$.
  Every back edge $(x', y') \in \diffset{E}{E(T)}$
  with $x', y' \not\in S$ is not a bridge in $G - S$
  since $x'y'$ and the $y'$--$x'$ path in $T$ form a cycle in $G - S$.
  Next, we consider tree edges in $G - S$.
  For any tree edge not on the $y$--$\parent(a)$-path in $T$,
  no covering back edge disappears by removing $S$ from $G$,
  % still exists in $G - S$,
  which means that such a tree edge is not a bridge in $G - S$.
  For any tree edge $(u, v)$ on the $y$--$\parent(a)$-path in $T$,
  by the assumption,
  there exists a back edge outgoing from a vertex in $T_v - S$
  that covers $(u,v)$,
  which means that such a tree edge is not a bridge in $G - S$.
  Thus, there is no bridge in $G - S$,
  and hence, $G - S$ is 2-edge-connected.
  \newline
  ($\text{(ii)}\Longleftrightarrow\text{(iii)}$)
  It is immediate when $S$ is a singleton $\{x\}$.
  When $S$ is a \maxTwoDegPath containing $x$,
  $x$ is the only vertex in $S$ that is a leaf of $T$.
  Thus, by \lemref{max2degpath_props}(i),
  only $x$ has back edges outgoing from it.
\end{proof}


% Here, we introduce the \emph{source-exclusive coverage set}~$\SrcExc_v$ of each vertex $v \in V$,
% defined to be the set of tree edge $e$ in $T$ such that
% $\CovSrc(e) = 1$ and $e$ is covered by
% only back edges outgoing from $v$.
% Or equivalently,
% if $e \in \SrcExc_v$,
% then $e$ is covered by back edges outgoing from $v$,
% and no back edge outgoing from any other vertex covers $e$.
% Any tree edge in $\SrcExc_v$ must be on the $\lbeTarget(v)$--$\parent(v)$ path in $T$.

% By \lemref{leaf_mrs},
% $S$ is a MinRS of $G$ if and only if,
% tree edges (u,v) with v notin S
% covering back edge (x1,y1), (x2,y2), ..., (xk,yk)
% → x1 != x or x2 != x or ... or xk != x

\begin{algorithm}[t]
  \caption{Computation of All MinRSs Containing a Leaf of a DFS-tree}
  \label{alg:leaf_mrs}
  \begin{algorithmic}[1]
    \Require A 2-edge-connected undirected graph $G=(V,E)$
    and its DFS-tree $T$ rooted at $r \in V$
    \Ensure All MinRSs of $G$ containing a leaf of $T$
    \State Compute $\CovMult(uv)$, $\CovDesc{1}(uv)$ and $\CovDesc{2}(uv)$
      for all tree edges~$uv$;\label{codeline:preproc_covdesc}
    \State \Initialise $\mathit{CS}_x := {}$ an array from vertices $x$ of $T$ to empty sets;
    \For{\textbf{each} tree edge $(u,v)$ in $E(T)$}
    \If{$(\CovDesc{1}(uv))^2 = \CovMult(uv)\cdot\CovDesc{2}(uv)$}
    \label{codeline:check_singleton}
    \State $x :=id^{-1}(\CovDesc{1}(uv) / \CovMult(uv))$;
    \State Append $(u,v)$ to $\mathit{CS}_x$ 
    \EndIf
      % \State
      % $X_{uv} :
      % \{x' \in V \mid \CovMult(uv)id(x')= (\CovDesc{1}(uv))^2 = \CovMult(uv)\cdot\CovDesc{2}(uv) \}$
      % % from $\CovDesc{1}(uv)$ and $\CovDesc{2}(uv)$;
      % \State Append $(u,v)$ to $\mathit{CS}_l$ \textbf{if} $X_{uv}$ is a singleton $\{l\}$ and $l$ is a leaf of $T$
      % \If{$X_{uv}$ is a singleton $\{x\}$ and $x$ is a leaf of $T$}
      %     \State Append $(u,v)$ to $\mathit{CS}_x$
      % \EndIf
    \EndFor;\label{codeline:build_leaf_links}
    \State \Initialise $\MY := {}$ an empty set;
    \For{\textbf{each} leaf $l$ of $T$}
      \State $S := \begin{cases}
      \{l\} & \text{if } \deg_G(l) \geq 3, \\
      \text{the vertex set of the \maxTwoDegPath containing $l$} & \text{otherwise};
      \end{cases}$
      \State $\textsc{is-MinRS} := \True$;
      \For{\textbf{each} tree edge $(u,v)$ in $\mathit{CS}_l$}
      \label{codeline:check_mrs_leaf}
        % \State $\textsc{is-MinRS} := \True$ \textbf{if} $u \not\in S$ and $v \not\in S$;
        % \Comment{You can break the loop here}
        \If{$v \not\in S$}
          \State $\textsc{is-MinRS} := \False$
          \Comment{You can break the loop here}
        \EndIf
      \EndFor;
      \State $\cupappend{\MY}{\{S\}}$ \textbf{if} $\textsc{is-MinRS}$
    \EndFor;
    \State \textbf{output} $\MY$
  \end{algorithmic}
\end{algorithm}
  

\begin{thm}
  \label{thm:leaf_mrs}
  Given a 2-edge-connected undirected graph $G=(V,E)$
  with $n=|V|$ and $m=|E|$,
  \algoref{leaf_mrs} computes all MinRSs of $G$ containing a leaf of a DFS-tree $T$
  in $O(n+m)$ time.
  % we can compute all MinRSs containing a leaf of a DFS-tree $T$
  % in $O(n+m)$ time.
\end{thm}
\begin{proof}
  Note that we assume that a graph has no parallel edges throughout this paper.
  If $|V| < 3$, i.e., $G$ consists of only one vertex,
  then $G$ has no MinRS.
  Thus, we assume that $|V| \geq 3$.
  As the given graph is 2-edge-connected,
  each tree edge is covered by at least one back edge.
  By \lemref{cov_info_use},
  for any tree edge $(u,v)$,
  line~\ref{codeline:check_singleton} checks whether
  all back edges covering $(u,v)$
  are outgoing from a single vertex $x$,
  and if so,
  $(u,v)$ is appended to $\mathit{CS}_x$.
  Therefore, the end of the for-loop in line~\ref{codeline:build_leaf_links},
  $\mathit{CS}_x$ contains all tree edges
  that are covered only by back edges outgoing from $x$.
  %
  Let $S \subsetneq V$ be 
  a nonempty proper subset of $V$ that
  forms a \maxTwoDegPath containing a leaf $l$ of $T$.
  By $\text{(i)}\Longleftrightarrow\text{(iii)}$ of  \lemref{leaf_mrs},
  $S$ is a MinRS of $G$ if and only if,
  for any tree edge $(u,v)$ with $v \not\in S$,
  there exists a back edge outgoing from other vertex than $l$
  that covers $(u,v)$;
  which is equivalent to that
  there is no tree edge $(u,v)$ in $\mathit{CS}_l$
  with $v \notin S$.
  We check this condition for each leaf $l$ of $T$
  in the for-loop starting from line~\ref{codeline:check_mrs_leaf}.
  We consider all leaves of $T$,
  which means that we can find all MinRSs containing a leaf of $T$.

  By \lemref{cov_info_agg},
  we can compute  $\CovMult$, $\CovDesc{1}$ and $\CovDesc{2}$
  for all tree edges in $O(n+m)$ time.
  For each tree edge $(u,v)$,
  we can compute the vertex $x$ such that
  all the back edge $(x',y')$ covering $(u,v)$ is outgoing from $x$ if such $x$ exists
  from $\CovMult(uv)$, $\CovDesc{1}(uv)$ and $\CovDesc{2}(uv)$
  by \lemref{cov_info_use}.
  For each leaf $l$ of $T$,
  we can check whether $S$ is a MinRS of $G$
  in time proportional to the size of $\mathit{CS}_l$.
  We can check all candidates $S$ in $O(n)$ time
  since the total size of $\mathit{CS}_l$
  for all leaves $l$ of $T$ is at most $O(n)$.
  Thus, the total time complexity is $O(n+m)$.
\end{proof}

% Max2deg Path MinRSs but all vertices are internal vertices of DFS-tree
\subsection{\MaxTwoDegPathCap MinRSs but all Internal Vertices}
\label{sec:max2deg_pathcap_mrs}
In this section, we consider MinRSs that consist of only internal vertices of a DFS-tree
and form a \maxTwoDegPath in $G$.
% In order to simplify discussions,
% we introduce the `suppression' for \maxTwoDegPaths.
To simplify discussions,
we introduce the \emph{suppression} for \maxTwoDegPaths.
First, we show that the two \neighbours of the endpoints of a \maxTwoDegPath
not containing the root or any leaf of a DFS-tree are distinct.
\begin{lem}
  \label{lem:suppression_valid}
  Let $G=(V,E)$ be a 2-edge-connected undirected graph with $|V|\geq 3$,
  and $P$ be a \maxTwoDegPath in $G$ that does not contain the root or any leaf of a DFS-tree $T$ of $G$.
  Then, $|N_G(V(P))| = 2$ holds.
\end{lem}
\begin{proof}
  Let $P = v_1 \to v_2 \to \cdots \to v_k$,
  and let $a, b$ denote the \neighbours of $v_1$ and $v_k$ not on $P$ if $P$ is not a singleton;
  if $P$ is a singleton, then let $a$ and $b$ denote the two \neighbours of the only vertex in $P$.
  We show that $a \neq b$ holds.
  By \lemref{max2degpath_props}(ii), 
  $P$ appears as the $v_1$--$v_k$ path in $T$,
  and without loss of generality,
  we assume that $v_1$ is an ancestor of $v_k$ in $T$.
  The DFS traversal visits in order $a, v_1, v_2, \ldots, v_k$.
  If $a = b$, then
  $v_k b$ is a back edge,
  which means that $b$ would be a leaf of $T$
  since $v_k$ has degree 2 and $b$ is its only neighbour other than $v_{k-1}$.
  This contradicts the assumption that $P$ does not contain any leaf of $T$.
\end{proof}

We fix a \maxTwoDegPath~$P = v_1 \to v_2 \to \cdots \to v_k$ in $G$, and let $\{a, b\} = N_G(V(P))$,
where $a \neq b$ holds by \lemref{suppression_valid}.
We define the \emph{suppression} of $P$
to be the graph $G_P$ obtained from $G$ by
removing all vertices of $P$
and adding a new edge $e_P=ab$,
which means that the new edge $e_P$ is not a self-loop.
Note that $G_P$ or the suppressed DFS-tree $T_P$ may have parallel edges even if $G$ has no parallel edges.

\begin{lem}
  \label{lem:dfs_tree_suppression}
  Let $G=(V,E)$ be a 2-edge-connected undirected graph with $|V|\geq 3$,
  $T$ be a DFS-tree of $G$ rooted at $r\in V$,
  and $P = v_1 \to v_2 \to \cdots \to v_k$ be a \maxTwoDegPath in $G$ that does not contain $r$ or any leaf of $T$;
  let $a$ and $b$ denote the \neighbours of $v_1$ and $v_k$ not on $P$.
  Then, $T_P$ is a DFS-tree of $G_P$ rooted at $r$,
  where $T_P$ is the tree obtained from $T$ by surpressing $P$.
  Moreover,
  $e_P$ is a tree edge in $T_P$ and
  $\CovMult(e_P)$ in $G_P$ equals any $\CovMult(uv)$ in $G$
  for a tree edge $(u,v)$ on the $a$--$b$ path in $T$.
\end{lem}
\begin{proof}
  Let $P = v_1 \to v_2 \to \cdots \to v_k$,
  and let $a, b$ denote the \neighbours of $v_1$ and $v_k$ not on $P$.
  By Lemmas~\ref{lem:max2degpath_props}(ii) and~\ref{lem:suppression_valid},
  $P$ appears as the $v_1$--$v_k$ path in $T$,
  and without loss of generality,
  we assume that $v_1$ is an ancestor of $v_k$ in $T$.
  In the DFS traversal that constructs $T$,
  the vertices in order $a, v_1, v_2, \ldots, v_k, b$ are visited one by one
  without visiting any other vertex in between.
  For DFS traversal on $G_P$,
  when we visit $a$,
  we can directly visit $b$ via the new edge $e_P = ab$
  without visiting any other vertex in between.
  Thus, $T_P$ is a DFS-tree of $G_P$ rooted at $r$.
  \newline
  By the construction of $T_P$,
  $e_P$ is a tree edge in $T_P$.
  Next, we show that
  $\CovMult(e_P)$ in $G_P$ equals any $\CovMult(uv)$ in $G$
  for a tree edge $(u,v)$ on the $a$--$b$ path in $T$.
  No vertex in $P$ is an endpoint of any back edge by \lemref{max2degpath_props}(i).
  Thus, any back edge in $G$ corresponds to a back edge in $G_P$ and vice versa.
  Moreover,
  a back edge covers $e_P$ in $G_P$
  if and only if it covers any tree edge $(u,v)$ on the $a$--$b$ path in $T$ in $G$.
  Thus, the cover multiplicities are equal.
\end{proof}

\begin{lem}
  \label{lem:2edgecut_covmult}
  Let $G=(V,E)$ be a 2-edge-connected undirected (multi)graph with $|V|\geq 3$,
  $T$ be a DFS-tree of $G$ rooted at $r\in V$, and
  $e=(u, v)$ be a tree edge in $T$.
  Then, $e$ belongs to some 2-edge-cut if and only if
  $\CovMult(e) = 1$ in $G$.
  Moreover, if $e$ belongs to some 2-edge-cut,
  then there exists a unique back edge $f$ such that
  $E_G(V(T_v)) = \{e,f\}$ holds.
\end{lem}
\begin{proof}
  Let $X := V(T_v)$.
  ($\Longrightarrow$)
  Let $\{e, f\}$ be a 2-edge-cut containing $e$ for some edge $f \neq e$.
  Here, $f$ must be a back edge;
  otherwise, i.e., if both $e$ and $f$ are tree edges,
  then $T - \{e, f\}$ has three connected components,
  and there would be at least one edge between any two of them
  in $G$,
  since every tree edge is covered by at least one back edge by the 2-edge-connectivity of $G$,
  which contradicts that $\{e, f\}$ is a 2-edge-cut.
  The removal of a back edge does not affect the connectivity within the connected components $X$ and $\diffset{V}{X}$ of $T - e$ in $G$,
  since any pair of vertices in each component
  are still connected via the tree edges in $T$.
  Thus, $f$ must disconnect $X$ from $\diffset{V}{X}$ in $G - e$,
  which means that $f$ is the only back edge covering $e$ in $G$.
  \newline
  ($\Longleftarrow$)
  Suppose that $\CovMult(e) = 1$.
  Let $f$ be the unique back edge covering $e$ in $G$.
  In $G - e$,
  $X$ and $\diffset{V}{X}$ are connected separately.
  Since $f$ is the only back edge covering $e$,
  $G - \{e, f\}$ is disconnected.
  Thus, $\{e, f\}$ is a 2-edge-cut containing $e$.
  \newline
  Moreover,
  if $e$ belongs to some 2-edge-cut,
  then the above argument shows that
  there exists a unique back edge $f \neq e$ such that
  $E_G(V(T_v)) = \{e,f\}$ holds.
\end{proof}



\begin{lem}
  \label{lem:max2degpath_mrs}
  Let $G=(V,E)$ be a 2-edge-connected undirected graph with $|V|\geq 3$,
  $T$ be a DFS-tree of $G$ rooted at $r\in V$,
  and $P$ be a \maxTwoDegPath in $G$ that does not contain $r$ or any leaf of $T$.
  The following are equivalent:
  \begin{enumerate}[label={\rm(\roman*)}]
    \item $P$ is a MinRS of $G$;
    \item $G_P - e_P$ is 2-edge-connected; and
    \item $\CovMult(e_P) \geq 2$ in $G_P$.
  \end{enumerate}
\end{lem}
\begin{proof}
  The equivalence between
  (i) and (ii) is immediate from the definition of MinRSs
  and the construction of $G_P$.
  \newline
  (ii) $\iff$ $G_P - e_P$ has no bridge $\iff$
  $G_P$ has no 2-edge-cut containing $e_P$
  $\iff$ (iii),
  where the last equivalence follows by \lemref{2edgecut_covmult}.
\end{proof}

\begin{thm}
  \label{thm:max2degpath_mrs}
  Given a 2-edge-connected undirected graph $G=(V,E)$
  with $n=|V|$ and $m=|E|$,
  and its DFS-tree $T$ rooted at $r \in V$,
  we can compute all MinRSs of $G$ consisting of only internal vertices of $T$
  that form a \maxTwoDegPath in $G$
  in $O(n+m)$ time.
\end{thm}
\begin{proof}
  Let $P$ be a \maxTwoDegPath in $G$ that does not contain $r$ or any leaf of $T$
  and $v$ denote the vertex in $P$ closest to $r$ in $T$.
  By Lemmas~\ref{lem:dfs_tree_suppression} and~\ref{lem:max2degpath_mrs},
  we can determine whether $P$ is a MinRS of $G$
  by checking whether $\CovMult(\parent(v)v) \geq 2$ in $G_P$.
  By \lemref{cov_info_agg},
  we can compute the cover multiplicities for all tree edges in $O(n+m)$ time.
\end{proof}

\subsection{Single Internal Vertex Degree At Least 3 MinRSs}
\label{sec:single_vertex_deg3_mrs}

\begin{lem}
  Let $G=(V,E)$ be a 2-edge-connected undirected graph with $|V|\geq 3$ and $v \in V$.
  The graph $G - v$ has a bridge $e$ if and only if
  there exists $X \subsetneq \diffset{V}{\{v\}}$ such that
  $E_G(X) = \{e\} \cup E_G(v, X)$. 
\end{lem}
\begin{proof}
  ($\Longrightarrow$)
  Let $X$ be the vertex set of a connected component of $(G - v) - e$, where $E_{G - v}(X) = \{e\}$ holds.
  Since $G$ is connected,
  all edges between $X$ and $\diffset{V}{X}$ in $G$ must be incident to $v$.
  Thus, $E_G(X) = \{e\} \cup E_G(v, X)$ holds.
  \newline
  ($\Longleftarrow$)
  Suppose that there exists $X \subsetneq \diffset{V}{\{v\}}$ such that $E_G(X) = \{e\} \cup E_G(v, X)$.
  In $G - v$, all edges between $X$ and $\diffset{V}{X}$ are removed except for $e$.
  Thus, $e$ is a bridge in $G - v$.
\end{proof}

TODO

\subsection{Reuse of DFS-tree and Covering Information}
\label{sec:reuse_dfs_cov}
TODO

%%%%%%%%%%%%%%%%%%%%%%%%%%%%%%%%%%%%%%%%%%%%%%%%%%%%%%%%%%%%
% \clearpage
% \input{unused_lemmas.tex}

% %%%%%%%%%%%%%%%%%%%%%%%%%%%%%%%%%%%%%%%%%%%%%%%%%%%%%%%%%%%%
\clearpage
\bibliographystyle{plainurl}
\bibliography{ref}
% \nocite{}

%%%%%%%%%%%%%%%%%%%%%%%%%%%%%%%%%%%%%%%%%%%%%%%%%%%%%%%%%%%%
%%%%% Appendix if any %%%%%

\end{document}


