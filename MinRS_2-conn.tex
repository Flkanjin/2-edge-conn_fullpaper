\documentclass[a4paper,11pt]{article}
%\documentclass[a4paper,11pt]{article}
%\jtwocolumn


\pagestyle{myheadings}\markboth{2-Edge-Connectivity: \today}{2-Edge-Connectivity: \today}

%%%%%%%%%%%%%%%%%%%%%%%%%%%%%%%%%%%%%%%%%%%%%%%%%%%%%%%%%%%%%%%%%%%%%%%%%%%%%%
% packages
\usepackage{graphicx}
\usepackage{epstopdf}
%\usepackage{tabularx}
\usepackage{latexsym}
%\usepackage{pstricks,psfrag}
\usepackage{amsmath,amssymb}
\usepackage{algorithm} %%% for comm. to N-sensei
\usepackage{algpseudocode} %%% for comm. to N-sensei
% \usepackage[ruled,vlined,linesnumbered]{algorithm2e}
\usepackage{tikz}
\usepackage{color}

%%%%%%%%%%%%%%%%%%%%%%%%%%%%%%%%%%%%%%%%%%%%%%%%%%%%%%%%%%%%%%%%%%%%%%%%%%%%%%
% newcommand
%\renewcommand{\mid}{:~}
\newcommand{\secref}[1]{Section~\ref{sec:#1}}
\newcommand{\tabref}[1]{Table~\ref{tab:#1}}
\newcommand{\figref}[1]{Figure~\ref{fig:#1}}
\newcommand{\corref}[1]{Corollary~\ref{cor:#1}}
\newcommand{\lemref}[1]{Lemma~\ref{lem:#1}}
\newcommand{\propref}[1]{Proposition~\ref{prop:#1}}
\newcommand{\thmref}[1]{Theorem~\ref{thm:#1}}

\newcommand{\bbZ}{{\mathbb Z}}
\newcommand{\bbR}{{\mathbb R}}
\newcommand{\MS}{{\mathcal S}}
\newcommand{\MC}{{\mathcal C}}

\newcommand{\vc}[1]{\mbox{\boldmath$ #1 $}}
\newcommand{\myhline}{
\noindent
\begin{tabular}{p{0.97\textwidth}}
\\\hline
\end{tabular}}
\newcommand{\tab}{\hspace*{1em}}



%%%%%%%%%%%%%%%%%%%%%%%%%%%%%%%%%%%%%%%%%%%%%%%%%%%%%%%%%%%%%%%%%%%%%%%%%%%%%%
% newtheorem
\newtheorem{cor}{Corollary}
\newtheorem{lem}{Lemma}
\newtheorem{prop}{Proposition}
\newtheorem{thm}{Theorem}

%%%%%%%%%%%%%%%%%%%%%%%%%%%%%%%%%%%%%%%%%%%%%%%%%%%%%%%%%%%%%%%%%%%%%%%%%%%%%%
% newenvironment
\newenvironment{proof}{\medskip
  \noindent{\scshape Proof:}}{\quad $\Box$\medskip}

\newenvironment{myframe}{\begin{trivlist}\item[]
    \hrule
    \hbox to \linewidth\bgroup
    \advance\linewidth by -30pt
    \hsize=\linewidth
    \vrule\hfill
    \vbox\bgroup
    \vskip15pt
    \def\thempfootnote{\arabic{mpfootnote}}
    \begin{minipage}{\linewidth}}{%
    \end{minipage}\vskip15pt
    \egroup\hfill\vrule
    \egroup\hrule
\end{trivlist}}

\algrenewcommand{\algorithmicrequire}{\textbf{Input: }}
\algrenewcommand{\algorithmicensure}{\textbf{Output: }}

\long\def\invis#1{}

%%%%%%%%%%%%%%%%%%%%%%%%%%%%%%%%%%%%%%%%%%%%%%%%%%%%%%%%%%%%%%%%%%%%%%%%%%%%%%
% style definitions
%
% following setting makes 3cm spaces for top and bottom, and
% 2.5cm spaces for left and right
%
                                         % default setting
\setlength{\oddsidemargin}{22pt}         % 62pt
\setlength{\evensidemargin}{22pt}        % 62pt
\setlength{\headheight}{12pt}            % 12pt
\setlength{\textheight}{662pt}           % 592pt
\setlength{\marginparsep}{10pt}          % 10pt
\setlength{\footskip}{30pt}              % 30pt
\setlength{\hoffset}{-13pt}              % 0pt
\setlength{\paperwidth}{597pt}           % 597pt
\setlength{\topmargin}{20pt}             % 20pt
\setlength{\headsep}{25pt}               % 25pt
\setlength{\textwidth}{427pt}            % 327pt
\setlength{\marginparwidth}{106pt}       % 106pt
\setlength{\marginparpush}{5pt}          % 5pt
\setlength{\voffset}{-37pt}              % 0pt
\setlength{\paperheight}{845pt}          % 845pt


% 1 inch = 2.54 cm = 72.27 pt

\renewcommand{\baselinestretch}{1.20}
% \setlength{\columnsep}{1.0cm}

\usepackage{xspace}
\newcommand{\lbe}{\mathit{lbe}}
\newcommand{\Cov}{\mathit{Cov}}
\newcommand{\CovMult}{\mathit{CovMult}}
\newcommand{\CovSrc}{\mathit{CovSrcMult}}

%for algo 1
\newcommand{\inCnt}{\mathit{in\_cnt}}

%%%%%%%%%%%%%%%%%%%%%%%%%%%%%%%%%%%%%%%%%%%%%%%%%%%%%%%%%%%%%%%%%%%%%%%%%%%%%%
% title information
\title{Simpler Computation of Minimal Removable Sets in 2-Edge-Connected Systems for Undirected Graphs}
\author{SHOTA, Kan\and HARAGUCHI, Kazuya}
%\date{}

%%%%%%%%%%%%%%%%%%%%%%%%%%%%%%%%%%%%%%%%%%%%%%%%%%%%%%%%%%%%%%%%%%%%%%%%%%%%%%
% document
\begin{document}
\maketitle

\section{Introduction}
\label{sec:intro}

%%%%%%%%%%%%%%%%%%%%%%%%%%%%%%%%%%%%%%%%%%%%%%%%%%%%%%%%%%%%
\section{Preliminaries}
\label{sec:prel}
Throughout the paper, we assume that a graph is simple and undirected.
% We also assume that a nonnegative integer $k$ is given. 

For a graph $G=(V,E)$ with a vertex set $V$
and an edge set $E$,
we may abbreviate an edge $\{u,v\} \in E$ into $uv$ (or equivalently $vu$)
for simplicity.
For a vertex $v \in V$, we denote by $N_G(v)$ the set of neighbors of $v$. 
For $S \subseteq V$, we denote by $G[S]$ the subgraph of $G$
induced by $S$.
We denote by $G-S$ the subgraph obtained by
removing $S$ and all edges to incident to $S$,
that is, $G - S = G[V-S]$.
If $S = \{v\}$,
then we simply write $G-v$ instead of $G - \{v\}$.


% DFS-Tree
A \emph{DFS-tree} of a connected undirected graph~$G = (V,E)$
is a spanning tree~$T$ obtained by performing a depth-first search (DFS) on $G$ .
An edge in $E$ is called a \emph{tree edge} if it belongs to $T$,
and a \emph{back edge} otherwise.
%
Let $T$ be rooted at a vertex $r \in V$.
For $u,v \in V$,
we say that $u$ is a \emph{ancestor} of $v$
if $u$ lies on the unique $r$--$v$-path in $T$.
If $u$ is an ancestor of $v$ and $u \neq v$,
then we say that $u$ is a \emph{proper ancestor} of $v$,
and that $v$ is a \emph{descendant} (resp., a \emph{proper descendant}) of $u$.
If $uv$ is a tree edge and $u$ is an ancestor of $v$,
then $u$ is the \emph{parent} of $v$;
we denote it by $\pi(v)=u$.

% DFS-index
During the DFS,
each vertex is assigned a unique integer called \emph{DFS-index},
which indicates the order of visiting time in the DFS.
For $v \in V$, we denote by $\mathit{dfsId}(v)$ the DFS-index of $v$.
An ancestor has a smaller DFS-index than any of its descendants.

% orientation of edges
We may orient edges based on the DFS-tree $T$:
for a tree edge $\pi(v)v$, we orient it from $\pi(v)$ to $v$; and
for a back edge $uv$, we orient it from the descendant to the ancestor.
We may write $(u,v)$ to indicate the orientation of edge $uv$
and refer to it as the \emph{outgoing edge} from $u$
and the \emph{incoming edge} to $v$.
% Subtree
For $v\in V$, we denote by $T_v$ the subtree of $T$
rooted at $v$.

%
We say that a tree edge $\pi(v)v$ is \emph{covered} by a back edge
$e=(x,y)$ if $x$ is a proper descendant of $v$ and
$y$ is a proper ancestor of $v$.
In this case, $e$ together with $\pi(v)v$ forms part of a cycle in $G$.
Note that every tree edge in a $2$-edge-connected graph is covered
by at least one back edge.
For a tree edge $uv$, we denote by
$
\Cov(uv)=\{(x,y)\in E \mid (x,y)\ \text{is a back edge covering}\ uv\}
$
the set of back edges that cover $uv$.
We refer to $\CovMult(uv) := |\Cov(uv)|$ as the \emph{cover multiplicity}
of the tree edge $uv$.
Furthermore, we define the \emph{cover-source multiplicity} of $uv$ to be
$
\CovSrc(uv)=\bigl|\{\,x \mid \text{There is } y \text{ s.t. } (x,y)\in \Cov(uv)\,\}\bigr|,
$
that is, the number of distinct descendants that serve as outgoing
endpoints of back edges covering~$uv$.

% longest back edge outgoing from v
For a vertex $v \in V$,
every vertex~$u$ with a back edge $vu$
lies on the unique $\pi(v)$--$r$ path in $T$.
Among such vertices~$u$,
let $u^*$ be the one with the smallest DFS-index,
that is, the one closest to $\pi(v)$ in $T$.
We call the back edge $vu^*$ the \emph{longest back edge outgoing from $v$}
and denote it by $\lbe(v)$ and $u^*$ by $\mathit{lbeTarget}(v)$.


%%%%%%%%%%%%%%%%%%%%%%%%%%%%%%%%%%%%%%%%%%%%%%%%%%%%%%%%%%%%

\section{Computation of Cover multiplicities}
\label{sec:cov_mult}
In this section, we present an $O(n+m)$-time algorithm
to compute the cover multiplicity and the cover-source multiplicity
for all tree edges in a 2-edge-connected undirected graph $G=(V,E)$
with $n=|V|$ and $m=|E|$.

% 二段階で行う
% 各 backedgeについて, その両端の頂点の値を更新する (ALgo2?)
% 各木辺 (u,v) について T_v の頂点についている値の総和を DFSにより集めていく (Algo1?)
The algorithm consists of two phases.
In the first phase, we process each back edge to update an auxiliary
value named \emph{cover-delta value} associated with its endpoints (see Algorithm~\ref{alg:preproc_backedge}).
In the second phase, we perform a DFS on $T$ to aggregate these values
along the tree and compute the multiplicity for each tree edge
(see Algorithm~\ref{alg:cov_mult}).
The details are described below.



% Algo I
\begin{algorithm}[t]
  \caption{Computation of Cover Multiplicities by DFS}
  \label{alg:cov_mult}
  \begin{algorithmic}[1]
    \Require A 2-edge-connected undirected graph $G=(V,E)$,
    its DFS-tree $T$ rooted at $r \in V$,
    % the value
    cover-delta values $\delta$ for all vertices in $V$, and
    a tree edge $(u, v)$ of $G$
    \Ensure The cover multiplicities of all tree edges in $T_v$
    \Procedure{CoverMultiplicity}{$G, T, \delta, (u,v)$}
    \State Initinialize $\CovMult(uv) := \delta(v)$;
    \For{\textbf{each} child $w$ of $v$ in $T$}
      \State Execute \Call{CoverMultiplicity}{$G, T, \delta, (v,w)$};
      \State $\CovMult(uv) := \CovMult(uv) + \CovMult(vw)$
    \EndFor;
    \State \textbf{return} $\CovMult(uv)$;
    \EndProcedure
  \end{algorithmic}
\end{algorithm}

% Algo II
\begin{algorithm}[t]
  \caption{Preprocessing Back Edges}
  \label{alg:preproc_backedge}
  \begin{algorithmic}[1]
    \Require A 2-edge-connected undirected graph $G=(V,E)$
    and its DFS-tree $T$ rooted at $r \in V$
    \Ensure The cover-delta values $\delta$ for all vertices in $V$
    \State Initialize $\delta(v) := 0$ for all $v \in V$;
    \For{\textbf{each} back edge $(x,y)$ of $G$}
    \label{codeline:backedge_loop}
      \State $\delta(x) := \delta(x) + 1$;
      \State $\delta(y) := \delta(y) - 1$
    \EndFor;
    \State \textbf{return} $\delta$
  \end{algorithmic}
\end{algorithm}


% % Algo全体
% The overall algorithm is as follows:
% \begin{enumerate}
% \item Execute Algorithm~\ref{alg:preproc_backedge} to compute
%   the cover-delta values for all vertices in $V$.
% \item Execute Algorithm~\ref{alg:cov_mult} with the tree edge
%   incident to the root $r$ to compute the cover multiplicities
%   for all tree edges in $T$.
% \end{enumerate}
If we want to compute the cover-source multiplicities,
we can modify Algorithm~\ref{alg:preproc_backedge}
% codeline:backedge_loop の部分を 全てのback edgeから各頂点のlongest outgoing back edge に限る
so that, in line \ref{codeline:backedge_loop} of Algorithm~\ref{alg:preproc_backedge},
we only consider the longest back edges outgoing from each vertex.

\begin{lem}
  \label{lem:cov_mult}
  Given a 2-edge-connected undirected graph $G=(V,E)$
  with $n=|V|$ and $m=|E|$,
  its DFS-tree $T$ rooted at $r \in V$, and 
  the cover-delta values $\delta$ for all vertices in $V$,
  Algorithm~\ref{alg:cov_mult} computes the cover multiplicities
  for all tree edges in $T_v$ in $O(|V(T_v)|)$ time
  for a tree edge $(u,v)$ of $G$.
\end{lem}
\begin{proof}
  We first show the case with $v$ being a leaf.
  In this case, since there is no child of $v$ in $T$,
  the algorithm returns $\CovMult(uv) = \delta(v)$
  that is equal to the number of back edges
  outgoing from $v$ that cover $uv$.
  This is correct by the definition of $\delta$.
  Next, we consider the case with $v$ being an internal vertex or the root.
  By the definition of $\delta$,
  the sum of $\delta$ values for all vertices in $T_v$
  is equal to the number of back edges
  outgoing from vertices in $T_v$ that cover $uv$,
  which is equal to $|\Cov(uv)|$.
  The algorithm aggregates the $\delta$ values
  for all vertices in $T_v$ by performing a DFS on $T_v$,
  and hence, it correctly computes $|\Cov(uv)|$.
  The time complexity is $O(|V(T_v)|)$,
  since the algorithm visits each vertex in $T_v$ exactly once.
\end{proof}

\begin{thm}
  \label{thm:cov_mult}
  Given a 2-edge-connected undirected graph $G=(V,E)$
  with $n=|V|$ and $m=|E|$,
  we can compute the cover multiplicity and the cover-source multiplicity
  for all tree edges in $G$ in $O(n+m)$ time.
\end{thm}
\begin{proof}
  We first perform a DFS on $G$ to obtain its DFS-tree $T$
  rooted at an arbitrary vertex $r \in V$ in $O(n+m)$ time.
  Next, we execute Algorithm~\ref{alg:preproc_backedge}
  to compute the cover-delta values for all vertices in $V$ in $O(n+m)$ time.
  Finally, we execute Algorithm~\ref{alg:cov_mult}
  with each tree edge $(r, v)$ incident to the root $r$
  to compute the cover multiplicities for all tree edges in $T$.
  By \lemref{cov_mult},
  this step takes $O(n)$ time.
  Thus, the total time complexity is $O(n+m)$.
\end{proof}


%%%%%%%%%%%%%%%%%%%%%%%%%%%%%%%%%%%%%%%%%%%%%%%%%%%%%%%%%%%%
\section{Mathematical Properties of Minimal Removable Sets}
\label{sec:mrs_props}
\subsection{DFS-Tree Leaf MinRSs}
\label{sec:leaf_mrs}
In this section, we consider MinRSs that consist of only leaves of a DFS-tree.

We first show a basic property of leaves in a DFS-tree
of a 2-edge-connected undirected graph.
\begin{lem}
  \label{lem:leaf_not_ap}
  Let $G=(V,E)$ be a 2-edge-connected undirected graph and
  $T$ be a DFS-tree of $G$ rooted at $r\in V$.
  Any leaf is not an articulation point of $G$.
\end{lem}
\begin{proof}
  Since $T$ is a spanning tree of $G$,
  every path in $T$ also exists in $G$.
  For any leaf $v\in V$ of $T$,
  $T - v$ is connected.
  Thus, $G - v$ is also connected,
  which means that $v$ is not an articulation point of $G$.
\end{proof}


%


% \begin{lem}
%   \label{lem:leaf_mrs}
%   Let $G=(V,E)$ be a 2-edge-connected undirected graph with $|V|\geq 3$ and,
%   $T$ be a DFS-tree of $G$ rooted at $r\in V$.
%   For each leaf $v$, $\{v\}$ is a MinRS of $G$ if and only if
%   the following two conditions hold:
%   \begin{itemize}
%   \item[\rm (i)] $\pi(v)$ has degree at least 3 in $G$; and
%   % \item[\rm (ii)] for any back edge $vu$,
%   % any tree edge on the $u,\pi(v)$-path in $T$ is covered by
%   % some back edge outgoing from other vertex than $v$. 
%   \item[\rm (ii)] let $\lbe(v)=(v,u^*)$,
%     then, for any tree edge $(x,y)$ on the $u^*,\pi(v)$-path in $T$,
%     there exists a back edge outgoing from a vertex in $T_y - v$
%     that covers $(x,y)$.
%   \end{itemize}
% \end{lem}
% \begin{proof}
%   ($\Longrightarrow$)
%   We show by contraposition.
%   Suppose that either (i) or (ii) does not hold.
%   If (i) does not hold, then
%   $\pi(v)$ has degree 1 in $G - v$,
%   which means that $G - v$ is not 2-edge-connected.
%   If (ii) does not hold,
%   then there exists a tree edge $(x,y)$ on the $u^*,\pi(v)$-path in $T$
%   that is not covered by any back edge outgoing from other vertex than $v$.
%   Since any vertex in $T_y - v$ cannot have a back edge to
%   a proper ancestor of $x$ in $G - v$,
%   which means that $xy$ is a bridge in $G - v$.
%   Thus, in either case, $G - v$ is not 2-edge-connected.
  
%   % \medskip
%   ($\Longleftarrow$)
%   We suffice to show that $G - v$ is 2-edge-connected.
%   By \lemref{leaf_not_ap},
%   $G - v$ is connected.
%   We show that there is no bridge in $G - v$.
%   Every back edge $(x, y)$ with $x, y \neq v$ is not a bridge in $G - v$,
%   since $xy$ and the $y,x$-path in $T$ form a cycle in $G - v$.
%   Next, we consider tree edges in $G - v$.
%   For any tree edge not on yhe $u^*,\pi(v)$-path in $T$,
%   the covering back edge still exists in $G - v$,
%   which means that such a tree edge is not a bridge in $G - v$.
%   For any tree edge $(x,y)$ on the $u^*,\pi(v)$-path in $T$,
%   by (ii),
%   there exists a back edge outgoing from a vertex in $T_y - v$
%   that covers $(x,y)$,
%   which means that such a tree edge is not a bridge in $G - v$.
%   Thus, there is no bridge in $G - v$,
%   and hence, $G - v$ is 2-edge-connected.
% \end{proof}


\begin{lem}
  \label{lem:leaf_mrs}
  Let $G=(V,E)$ be a 2-edge-connected undirected graph with $|V|\geq 3$ and,
  $T$ be a DFS-tree of $G$ rooted at $r\in V$.
  For each leaf $v$, let $\lbe(v)=(v,u^*)$.
  Then, $\{v\}$ is a MinRS of $G$ if and only if,
  for any tree edge $(x,y)$ on the $u^*,\pi(v)$-path in $T$,
  there exists a back edge outgoing from a vertex in $T_y - v$
  that covers $(x,y)$.
\end{lem}
\begin{proof}
  ($\Longrightarrow$)
  We show by contraposition.
  Suppose that there exists a tree edge $(x,y)$ on the $u^*,\pi(v)$-path in $T$
  that is not covered by any back edge outgoing from a vertex  other than $v$.
  Since any vertex in $T_y - v$ cannot have a back edge to
  a proper ancestor of $x$ in $G - v$,
  which means that $xy$ is a bridge in $G - v$.
  Thus, $G - v$ is not 2-edge-connected.
  
  % \medskip
  ($\Longleftarrow$)
  We suffice to show that $G - v$ is 2-edge-connected.
  By \lemref{leaf_not_ap},
  $G - v$ is connected.
  We show that there is no bridge in $G - v$.
  Every back edge $(x, y)$ with $x, y \neq v$ is not a bridge in $G - v$,
  since $xy$ and the $y,x$-path in $T$ form a cycle in $G - v$.
  Next, we consider tree edges in $G - v$.
  For any tree edge not on yhe $u^*,\pi(v)$-path in $T$,
  the covering back edge still exists in $G - v$,
  which means that such a tree edge is not a bridge in $G - v$.
  For any tree edge $(x,y)$ on the $u^*,\pi(v)$-path in $T$,
  by the assumption,
  there exists a back edge outgoing from a vertex in $T_y - v$
  that covers $(x,y)$,
  which means that such a tree edge is not a bridge in $G - v$.
  Thus, there is no bridge in $G - v$,
  and hence, $G - v$ is 2-edge-connected.
\end{proof}

As in Algorithm~\ref{alg:treeedge_to_backedge},
for any leaf $v$ of $T$,
we can store the set of tree edges $e$ in $T_v$ such that
$\CovSrc(e) = 1$ and $e$ is covered by back edges outgoing from $v$.
% ALGO
\begin{algorithm}[t]
  \caption{Computation of the Set of Tree Edges with Cover-Source Multiplicity 1 for Each Leaf}
  \label{alg:treeedge_to_backedge}
  \begin{algorithmic}[1]
    \Require A 2-edge-connected undirected graph $G=(V,E)$,
    its DFS-tree $T$ rooted at $r \in V$, 
    and the cover-source multiplicities $\CovSrc$ for all tree edges in $T$, and
    a tree edge $(u, v)$ of $G$
    \Ensure For each leaf $v$ of $T$, the set $\mathit{BE}_v$ of tree edges  with cover-source multiplicity 1
    that are covered by back edges outgoing from $v$;
    and leaf $l$ to be considered (possibly \textsc{Nil})
    \Procedure{TreeEdgeToBackEdge}{$G, T, \CovSrc, (u,v)$}
    \State $l := \textsc{Nil}$;
    \If{$v$ is a leaf}
      \State $l := v$
    \EndIf;
    \For{\textbf{each} child $w$ of $v$ in $T$}
      \State Execute \Call{TreeEdgeToBackEdge}{$G, T, \CovSrc, (v,w)$};
      \State Let $l'$ be the returned leaf;
      \If{$l' \neq \textsc{Nil}$ and [$l = \textsc{Nil}$ or $ \mathit{dfsId}(\mathit{lbeTarget}(l')) < \mathit{dfsId}(\mathit{lbeTarget}(l))$]}
        \State $l := l'$;
      \EndIf
    \EndFor
    

    \State $\mathit{BE}_l := \mathit{BE}_l \cup \{uv\}$ \textbf{if} $\CovSrc(uv) = 1$; \Comment{$l$ is not \textsc{Nil} here if this line is executed}
    \EndProcedure
  \end{algorithmic}
\end{algorithm}

\begin{lem}
  \label{lem:treeedge_to_backedge}
  Given a 2-edge-connected undirected graph $G=(V,E)$
  with $n=|V|$ and $m=|E|$,
  its DFS-tree $T$ rooted at $r \in V$, and 
  the cover-source multiplicities $\CovSrc$ for all tree edges in $T$,
  Algorithm~\ref{alg:treeedge_to_backedge} computes,
  for each leaf $v$ of $T$,
  the set $\mathit{BE}_v$ of tree edges $e$ in $T_v$ such that
  $\CovSrc(e) = 1$ and $e$ is covered by back edges outgoing from $v$
  in $O(n)$ time.
\end{lem}
\begin{proof}
  TODO
\end{proof}

\begin{thm}
  \label{thm:leaf_mrs}
  Given a 2-edge-connected undirected graph $G=(V,E)$
  with $n=|V|$ and $m=|E|$,
  we can compute all MinRSs consisting of only leaves of a DFS-tree
  in $O(n+m)$ time.
\end{thm}
\begin{proof}
  As the given graph is 2-edge-connected,
  each tree edge is covered by at least one back edge.
  By \lemref{leaf_mrs}, $\{v\}$ is a MinRS of $G$ if and only if
  for any tree edge is covered by back edges outgoing from vertices other than $v$,
  which means that $\CovSrc(e) \geq 2$.
  Thus, we can check whether $\{v\}$ is a MinRS of $G$
  by checking whether $\mathit{BE}_v$ is empty or not.
  By \thmref{cov_mult},
  we can compute the cover-source multiplicities for all tree edges
  in $O(n+m)$ time.
  By \lemref{treeedge_to_backedge},
  we can compute the sets $\mathit{BE}_v$ for all leaves $v$ of $T$
  in $O(n)$ time.
  Thus, we can compute all MinRSs consisting of only leaves of $T$
  in $O(n+m)$ time.
\end{proof}

\subsection{Maximal 2-deg Path MiRSs}
\label{sec:2deg_path_mrs}
TODO

\subsection{Single Vertex Degree at Least 3 MinRSs}
\label{sec:single_vertex_deg3_mrs}
TODO

\subsection{Reuse of DFS-tree and Covering Information}
\label{sec:reuse_dfs_cov}
TODO

%%%%%%%%%%%%%%%%%%%%%%%%%%%%%%%%%%%%%%%%%%%%%%%%%%%%%%%%%%%%
% \clearpage
% \input{unused_lemmas.tex}

\end{document}


